\documentclass[12pt,a4paper,titlepage]{article}
\usepackage[T1]{fontenc}
\usepackage[utf8]{inputenc}
\usepackage[polish]{babel}
\usepackage{graphicx}
\usepackage{caption}
\title{Bazy Danych I - Projekt Semestralny}
\author{Dariusz Mydlarz, Michał Sitko}
\date{\today}
\begin{document}

\maketitle
\newpage


\section{Struktura bazy}
\subsection{Schemat}
\begin{center}
	\includegraphics[width=1.00\textwidth]{bazyNew.png}
	\captionof{figure}{Schemat bazy danych}
	\label{fig:schemat}
\end{center}


\subsection{Tabele}
Pola pogrubione są polami wymaganymi.
% segment tabela
\paragraph{member}
przechowuje podstawowe informacje o kliencie
\subparagraph{pola:}
\textbf{memberID (PK), firstname, lastname, phone, active}, email
\subparagraph{sprawdzanie poprawności danych:}
\begin{itemize}
	\item phone, \textit{długość pola musi być równa 11}
\end{itemize}
% ends tabela

\newpage

% segment tabela
\paragraph{juvenile} przechowuje szczegółowe informacje o niepełnoletnim kliencie
\subparagraph{pola:}
\textbf{memberID (PK, FK:\underline{member}), adult\_memberID (FK),} \\ \textbf{birthDate}
\subparagraph{sprawdzanie poprawności danych:}
\begin{itemize}
	\item birthDate, \textit{musi być mniejsza od daty wprowadzania i większa od tej samej daty pomniejszonej o 18 lat }
\end{itemize}
% ends tabela

% segment tabela
\paragraph{adult} przechowuje szczegółowe informacje o dorosłym kliencie
\subparagraph{pola:}
\textbf{memberID (PK, FK:\underline{member}), street, homeNo, city, state, zip, expirationDate}, flatNo
\subparagraph{sprawdzanie poprawności danych:}
\begin{itemize}
	\item expirationDate, \textit{data wygaśnięcia musi być większa od aktualnej daty}
	\item zip, \textit{długość pola musi być równa 5}
\end{itemize}
\subparagraph{wartości domyślne:}
\begin{itemize}
	\item expirationDate, \textit{aktualna data + 5 lat}
\end{itemize}
% ends tabela

% segment tabela
\paragraph{reservation} przechowuje informacje o złożonych rezerwacjach
\subparagraph{pola:}
\textbf{memberID (PK, FK:\underline{member}), mediumID (PK, FK:\underline{item}), filmID (PK, FK:\underline{member}), logDate, acceptDate}
\subparagraph{unikalne pola, zestawy pól:}
\begin{itemize}
	\item (memberID, filmID, mediumID), \textit{unikalna trójka: klient, film i nośnik filmu}
\end{itemize}
\subparagraph{sprawdzanie poprawności danych:}
\begin{itemize}
	\item logDate, \textit{nie może być inna niż aktualna data}
	\item acceptDate, \textit{musi być większa niż logDate}
\end{itemize}
\subparagraph{wartości domyślne:}
\begin{itemize}
	\item logDate, \textit{aktualna data}
\end{itemize}
% ends tabela

% segment tabela
\paragraph{film} przechowuje informacje o filmach w zbiorach wypożyczalni
\subparagraph{pola:}
\textbf{filmID (PK), filmPrice, title, director}, description
\subparagraph{unikalne pola, zestawy pól:}
\begin{itemize}
	\item (title, director), \textit{unikalna para: nazwa filmu i reżyser}
\end{itemize}
% ends tabela

% segment tabela
\paragraph{label} przechowuje informacje o etykietach dla filmów (nowość, promocja, hit)
\subparagraph{pola:}
\textbf{labelID (PK), labelName, priceMultiply, loanPeriod}
\subparagraph{unikalne pola, zestawy pól:}
\begin{itemize}
	\item labelName, \textit{unikalna nazwa etykiety (nowość, promocja, hit)}
\end{itemize}
\subparagraph{wartości domyślne:}
\begin{itemize}
	\item priceMultiply, \textit{1.0}
\end{itemize}
% ends tabela

% segment tabela
\paragraph{film\_and\_label} tabela łącznikowa między filmami, a etykietami
\subparagraph{pola:}
\textbf{filmID (PK, FK:\underline{film}), labelID (PK, FK:\underline{label}), expirationDate}
\subparagraph{unikalne pola, zestawy pól:}
\begin{itemize}
	\item (filmID, labelID), \textit{unikalna para: film i etykieta}
\end{itemize}
\subparagraph{sprawdzanie poprawności danych:}
\begin{itemize}
	\item expirationDate, \textit{musi być większa od daty wprowadzania}
\end{itemize}
% ends tabela

\newpage

% segment tabela
\paragraph{reservation\_and\_film\_and\_label}
\subparagraph{}
Tabela łącznikowa pomiędzy rezerwacją, a filmem z etykietą. Idea tabeli jest następująca: klient zarezerwował film w trakcie trwania jakiejś promocji (np. świątecznej).
Jednak film aktualnie jest niedostępny. Mimo to, nawet po zakończeniu promocji, będzie mógł wypożyczyć film zgodnie z cennikiem jaki oferowała trwająca podczas rezerwacji promocja.
I właśnie informacje o tym przechowujemy w tej tabeli.
\subparagraph{pola:}
\textbf{memberID (PK, FK:\underline{member}), filmID (PK, FK:\underline{film}),} \\\textbf{labelID (PK, FK:\underline{label}), mediumID(PK, FK:\underline{medium})}
\subparagraph{unikalne pola, zestawy pól:}
\begin{itemize}
	\item (memberID, labelID, filmID, mediumID), \textit{unikalna czwórka: klient, film, etykieta, nośnik filmu}
\end{itemize}
% ends tabela

% segment tabela
\paragraph{category} przechowuje informacje o kategoriach filmów
\subparagraph{pola:}
\textbf{categoryID (PK), categoryName}
\subparagraph{unikalne pola, zestawy pól:}
\begin{itemize}
	\item categoryName, \textit{unikalna nazwa kategorii}
\end{itemize}
% ends tabela

% segment tabela
\paragraph{film\_and\_category} tabela łącznikowa między filmami, a kategoriami
\subparagraph{pola:}
\textbf{filmID (PK, FK:\underline{film}), categoryID (PK, FK:\underline{category})}
\subparagraph{unikalne pola, zestawy pól:}
\begin{itemize}
	\item (filmID, categoryID), \textit{unikalna para: film i kategoria}
\end{itemize}
% ends tabela

\newpage

% segment tabela
\paragraph{medium} przechowuje informacje o nośnikach filmów
\subparagraph{pola:}
\textbf{mediumID (PK), mediumName, priceMultiply}
\subparagraph{unikalne pola, zestawy pól:}
\begin{itemize}
	\item mediumName, \textit{unikalna nazwa nośnika filmu (dvd, blu-ray)}
\end{itemize}
\subparagraph{wartości domyślne:}
\begin{itemize}
	\item priceMultiply, \textit{1.0}
\end{itemize}
% ends tabela

% segment tabela
\paragraph{item} tabela łącznika między filmem, a nośnikiem
\subparagraph{pola:}
\textbf{filmID (PK, FK:\underline{film}), mediumID (PK, FK:\underline{medium})}
\subparagraph{unikalne pola, zestawy pól:}
\begin{itemize}
	\item (filmID, mediumID), \textit{unikalna para: film i nośnik filmu}
\end{itemize}
% ends tabela

% segment tabela
\paragraph{copy} przechowuje informacje o wszystkich kopiach filmów
\subparagraph{pola:}
\textbf{copyID (PK), filmID (FK:\underline{film}), mediumID (FK:\underline{medium}), onLoan}
% ends tabela

% segment tabela
\paragraph{loan} przechowuje informacje o wszystkich aktualnych wypożyczeniach
\subparagraph{pola:}
\textbf{copyID (PK, FK:\underline{copy}), memberID (FK:\underline{member})}, \\\textbf{outDate, dueDate}
\subparagraph{unikalne pola, zestawy pól:}
\begin{itemize}
	\item (copyID, memberID), \textit{unikalna para: kopia filmu i klient}
\end{itemize}
\subparagraph{sprawdzanie poprawności danych:}
\begin{itemize}
	\item dueDate, \textit{musi być większa od outDate}
\end{itemize}
\subparagraph{wartości domyślne:}
\begin{itemize}
	\item outDate, \textit{aktualna data}
\end{itemize}
% ends tabela

% segment tabela
\paragraph{loanhist} archiwum wypożyczeń
\subparagraph{pola:}
\textbf{outDate (PK), copyID (PK, FK:\underline{copy}), memberID (FK:\underline{member}), dueDate, inDate}, fineAssessed, finePaid, fineWaived, remarks
\subparagraph{unikalne pola, zestawy pól:}
\begin{itemize}
	\item (outDate, copyID), \textit{unikalna para: data wydania i numer kopii}
\end{itemize}
\subparagraph{sprawdzanie poprawności danych:}
\begin{itemize}
	\item finePaid, \textit{musi być mniejsza lub równa różnicy kwot fineAssessed i fineWaived}
\end{itemize}
\subparagraph{wartości domyślne:}
\begin{itemize}
	\item inDate, \textit{aktualna data}
\end{itemize}
% ends tabela

% #########################################################################################################
%	PROCEDURY, TRIGGERY, WIDOKI
% #########################################################################################################

\newpage
\section{Procedury, Triggery, Widoki}


\subsection{Procedury}
% segment procedura
\paragraph{deleteMember}
Usuwa użytkownika. Przy usuwaniu rodzica usuwa wszystkie jego dzieci.
\subparagraph{Argumenty:}
\begin{itemize}
	\item @member\_id, \textit{id użytkownika}
\end{itemize}
% ends procedura

% segment procedura
\paragraph{setMemberActive}
Ustawia wartość pola active dla użytkownika. Propagacja na dzieci poprzez trigger tr\_updateMember.
\subparagraph{Argumenty:}
\begin{itemize}
	\item @member\_id, \textit{id użytkownika}
	\item @active (0, 1), \textit{0 gdy chcemy ustawić na nieaktywnego, 1 gdy na aktywnego}
\end{itemize}
% ends procedura

% segment procedura
\paragraph{insertLoan}
Dodaje rekord do tabeli wypożyczeń. Zwraca zniżkę w procentach dla klientów, którzy w przeciągu ostatnich 3 miesięcy pożyczyli co najmniej 3 filmy.
Zablokuje możliwość dodania rekordu gdy klient ma już wypożyczone 6 filmów lub gdy przetrzymuje chociaż jeden film lub gdy jest nieaktywny.
\subparagraph{Argumenty:}
\begin{itemize}
	\item @member\_id, \textit{id użytkownika}
	\item @copy\_id, \textit{id kopii filmu}
\end{itemize}
\subparagraph{Wartość zwracana:}
\begin{itemize}
	\item @discount, \textit{rabat dla klienta}
\end{itemize}
% ends procedura

% segment procedura
\paragraph{insertReservation}
Dodaje rekord do tabeli rezerwacji. Zablokuje możliwość dodania rekordu gdy klient ma już wypożyczone 6 filmów lub gdy przetrzymuje chociaż jeden film lub gdy jest nieaktywny.
\subparagraph{Argumenty:}
\begin{itemize}
	\item @member\_id, \textit{id użytkownika}
	\item @film\_id, \textit{id filmu}
	\item @medium\_id, \textit{id nośnika filmu}
\end{itemize}
% ends procedura

% segment procedura
\paragraph{insertAdult}
Dodaje dorosłego klienta do tabel \underline{member} oraz \underline{adult}.
\subparagraph{Argumenty:}
\begin{itemize}
	\item @firstname, \textit{imię}
	\item @lastname, \textit{nazwisko}
	\item @phone, \textit{numer telefonu}
	\item @email, \textit{adres e-mail}
	\item @street, \textit{nazwa ulicy}
	\item @homeNo, \textit{numer domu}
	\item @flatNo, \textit{numer mieszkania}
	\item @city, \textit{miejscowość zamieszkania}
	\item @state, \textit{województwo/land/stan/region}
	\item @zip, \textit{kod kreskowy}
\end{itemize}
% ends procedura

% segment procedura
\paragraph{insertJuvenile}
Dodaje niepełnoletniego klienta do tabel \underline{member} oraz \underline{juvenile}.
\subparagraph{Argumenty:}
\begin{itemize}
	\item @firstname, \textit{imię}
	\item @lastname, \textit{nazwisko}
	\item @phone, \textit{numer telefonu}
	\item @email, \textit{adres e-mail}
	\item @adult\_id, \textit{identyfikator rodzica}
	\item @birthdate, \textit{data urodzenia}
\end{itemize}
% ends procedura



% #########################################################################################################
%	TRIGGERY
% #########################################################################################################

\subsection{Triggery}
% segment trigger
\paragraph{tr\_deleteLoan}
Usuwany rekod z tabeli \underline{loan} trafia do tabeli \underline{loanhist}. Jeśli data oddania została przekroczona, automatycznie naliczana jest kara.
% ends trigger

% segment trigger
\paragraph{tr\_updateMember}
Zmienia stan użytkownika.
% ends trigger

% segment trigger
\paragraph{tr\_insertViewMembers}
Dodaje użytkowników korzystając z widoku \underline{view_members}.
% ends trigger



% #########################################################################################################
%	WIDOKI
% #########################################################################################################

\subsection{Widoki}

% segment widok
\paragraph{view\_members}
Wyświetla wszystkie informacje o wszystkich klientach w jednej tabeli.
% ends widok

% segment widok
\paragraph{view\_films}
Wyświetla wszystkie informacje o wszystkich filmach w jednej tabeli.
% ends widok

% segment widok
\paragraph{view\_topFilms}
Wyświetla listę 10 najczęściej wypożyczanych filmów w ostatnim miesiącu.
% ends widok

% segment widok
\paragraph{view\_topClients}
Wyświetla listę 10 najlepszych klientów z ostatniego miesiąca.
% ends widok

\end{document}
