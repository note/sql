\documentclass[12pt,a4paper,titlepage]{article}
\usepackage[T1]{fontenc}
\usepackage[utf8]{inputenc}
\usepackage[polish]{babel}
\usepackage{graphicx}
\usepackage{caption}
\title{Bazy Danych I - Projekt Semestralny}
\author{Dariusz Mydlarz, Michał Sitko}
\date{\today}
\begin{document}

\maketitle
\newpage


\section{Struktura bazy}
\subsection{Schemat}
\begin{center}
	%\includegraphics[width=1.00\textwidth]{bazyNew.png}
	\captionof{figure}{Schemat bazy danych}
	\label{fig:schemat}
\end{center}


\subsection{Tabele}
Pola pogrubione są polami wymaganymi.
% segment tabela
\paragraph{member}
\subparagraph{pola:}
\textbf{memberID (PK), firstname, lastname, phone, active}, email
\subparagraph{sprawdzanie poprawności danych:}
\begin{itemize}
	\item phone, \textit{długość pola musi być równa 11}
\end{itemize}
% ends tabela

\newpage

% segment tabela
\paragraph{juvenile}
\subparagraph{pola:}
\textbf{memberID (PK, FK:\underline{member}), adult\_memberID (FK),} \\ \textbf{birthDate}
\subparagraph{sprawdzanie poprawności danych:}
\begin{itemize}
	\item birthDate, \textit{musi być mniejsza od daty wprowadzania i większa od tej samej daty pomniejszonej o 18 lat }
\end{itemize}
% ends tabela

% segment tabela
\paragraph{adult}
\subparagraph{pola:}
\textbf{memberID (PK, FK:\underline{member}), street, homeNo, city, state, zip, expirationDate}, flatNo
\subparagraph{sprawdzanie poprawności danych:}
\begin{itemize}
	\item expirationDate, \textit{data wygaśnięcia musi być większa od aktualnej daty}
	\item zip, \textit{długość pola musi być równa 5}
\end{itemize}
\subparagraph{wartości domyślne:}
\begin{itemize}
	\item expirationDate, \textit{aktualna data + 5 lat}
\end{itemize}
% ends tabela

% segment tabela
\paragraph{reservation}
\subparagraph{pola:}
\textbf{memberID (PK, FK:\underline{member}), mediumID (PK, FK:\underline{item}), filmID (PK, FK:\underline{member}), logDate, acceptDate}
\subparagraph{unikalne pola, zestawy pól:}
\begin{itemize}
	\item (memberID, filmID, mediumID), \textit{unikalna trójka: klient, film i nośnik filmu}
\end{itemize}
\subparagraph{sprawdzanie poprawności danych:}
\begin{itemize}
	\item logDate, \textit{nie może być inna niż aktualna data}
	\item acceptDate, \textit{musi być większa niż logDate}
\end{itemize}
\subparagraph{wartości domyślne:}
\begin{itemize}
	\item logDate, \textit{aktualna data}
\end{itemize}
% ends tabela

% segment tabela
\paragraph{film}
\subparagraph{pola:}
\textbf{filmID (PK), filmPrice, title, director}, description
\subparagraph{unikalne pola, zestawy pól:}
\begin{itemize}
	\item (title, director), \textit{unikalna para: nazwa filmu i reżyser}
\end{itemize}
% ends tabela

% segment tabela
\paragraph{label}
\subparagraph{pola:}
\textbf{labelID (PK), labelName, priceMultiply, loanPeriod}
\subparagraph{unikalne pola, zestawy pól:}
\begin{itemize}
	\item labelName, \textit{unikalna nazwa etykiety (nowość, promocja, hit)}
\end{itemize}
\subparagraph{wartości domyślne:}
\begin{itemize}
	\item priceMultiply, \textit{1.0}
\end{itemize}
% ends tabela

% segment tabela
\paragraph{film\_and\_label}
\subparagraph{pola:}
\textbf{filmID (PK, FK:\underline{film}), labelID (PK, FK:\underline{label}), expirationDate}
\subparagraph{unikalne pola, zestawy pól:}
\begin{itemize}
	\item (filmID, labelID), \textit{unikalna para: film i etykieta}
\end{itemize}
\subparagraph{sprawdzanie poprawności danych:}
\begin{itemize}
	\item expirationDate, \textit{musi być większa od daty wprowadzania}
\end{itemize}
% ends tabela

\newpage

% segment tabela
\paragraph{reservation\_and\_film\_and\_label}
\subparagraph{}
Tabela łącznikowa pomiędzy rezerwacją, a filmem z etykietą. Idea tabeli jest następująca: klient zarezerwował film w trakcie trwania jakiejś promocji (np. świątecznej).
Jednak film aktualnie jest niedostępny. Mimo to, nawet po zakończeniu promocji, będzie mógł wypożyczyć film zgodnie z cennikiem jaki oferowała trwająca podczas rezerwacji promocja.
I właśnie informacje o tym przechowujemy w tej tabeli.
\subparagraph{pola:}
\textbf{memberID (PK, FK:\underline{member}), filmID (PK, FK:\underline{film}),} \\\textbf{labelID (PK, FK:\underline{label}), mediumID(PK, FK:\underline{medium})}
\subparagraph{unikalne pola, zestawy pól:}
\begin{itemize}
	\item (memberID, labelID, filmID, mediumID), \textit{unikalna czwórka: klient, film, etykieta, nośnik filmu}
\end{itemize}
% ends tabela

% segment tabela
\paragraph{category}
\subparagraph{pola:}
\textbf{categoryID (PK), categoryName}
\subparagraph{unikalne pola, zestawy pól:}
\begin{itemize}
	\item categoryName, \textit{unikalna nazwa kategorii}
\end{itemize}
% ends tabela

% segment tabela
\paragraph{film\_and\_category}
\subparagraph{pola:}
\textbf{filmID (PK, FK:\underline{film}), categoryID (PK, FK:\underline{category})}
\subparagraph{unikalne pola, zestawy pól:}
\begin{itemize}
	\item (filmID, categoryID), \textit{unikalna para: film i kategoria}
\end{itemize}
% ends tabela

\newpage

% segment tabela
\paragraph{medium}
\subparagraph{pola:}
\textbf{mediumID (PK), mediumName, priceMultiply}
\subparagraph{unikalne pola, zestawy pól:}
\begin{itemize}
	\item mediumName, \textit{unikalna nazwa nośnika pamięci (dvd, blu-ray)}
\end{itemize}
\subparagraph{wartości domyślne:}
\begin{itemize}
	\item priceMultiply, \textit{1.0}
\end{itemize}
% ends tabela

% segment tabela
\paragraph{item}
\subparagraph{pola:}
\textbf{filmID (PK, FK:\underline{film}), mediumID (PK, FK:\underline{medium})}
\subparagraph{unikalne pola, zestawy pól:}
\begin{itemize}
	\item (filmID, mediumID), \textit{unikalna para: film i nośnik filmu}
\end{itemize}
% ends tabela

% segment tabela
\paragraph{copy}
\subparagraph{pola:}
\textbf{copyID (PK), filmID (FK:\underline{film}), mediumID (FK:\underline{medium}), onLoan}
% ends tabela

% segment tabela
\paragraph{loan}
\subparagraph{pola:}
\textbf{copyID (PK, FK:\underline{copy}), memberID (FK:\underline{member})}, \\\textbf{outDate, dueDate}
\subparagraph{unikalne pola, zestawy pól:}
\begin{itemize}
	\item (copyID, memberID), \textit{unikalna para: kopia filmu i klient}
\end{itemize}
\subparagraph{sprawdzanie poprawności danych:}
\begin{itemize}
	\item outDate, \textit{musi być równa dacie wprowadzenia rekordu}
	\item dueDate, \textit{musi być większa od outDate}
\end{itemize}
\subparagraph{wartości domyślne:}
\begin{itemize}
	\item outDate, \textit{aktualna data}
\end{itemize}
% ends tabela

% segment tabela
\paragraph{loanhist}
\subparagraph{pola:}
\textbf{outDate (PK), copyID (PK, FK:\underline{copy}), memberID (FK:\underline{member}), dueDate, inDate}, fineAssessed, finePaid, fineWaived, remarks
\subparagraph{unikalne pola, zestawy pól:}
\begin{itemize}
	\item (outDate, copyID), \textit{unikalna para: data wydania i numer kopii}
\end{itemize}
\subparagraph{sprawdzanie poprawności danych:}
\begin{itemize}
	\item inDate, \textit{musi być równa dacie wprowadzenia rekordu}
	\item finePaid, \textit{musi być mniejsza lub równa różnicy kwot fineAssessed i fineWaived}
\end{itemize}
\subparagraph{wartości domyślne:}
\begin{itemize}
	\item inDate, \textit{aktualna data}
\end{itemize}
% ends tabela

\section{Procedury}
\subsection{Procedury modyfikujące}
\subsection{Procedury niemodyfikujące}

\end{document}
